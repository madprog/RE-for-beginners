\newcommand{\NormalScale}{0.66} % FIXME?
\ifdefined\ebook
\documentclass[a5paper,oneside]{book}
\newcommand{\FigScale}{0.4}
\else
\documentclass[a4paper,oneside]{book}
\newcommand{\FigScale}{\NormalScale} 
\fi

\usepackage{fontspec}
% fonts
\setmonofont{Droid Sans Mono}
\setmainfont[Ligatures=TeX]{PT Sans}
\usepackage{polyglossia}
\defaultfontfeatures{Scale=MatchLowercase} % ensure all fonts have the same 1ex
\usepackage{ucharclasses}
\usepackage{csquotes}

\ifdefined\ENGLISH
\setmainlanguage{english}
\setotherlanguage{russian}
\fi

\ifdefined\RUSSIAN
\setmainlanguage{russian}
\newfontfamily\cyrillicfonttt{Droid Sans Mono}
\setotherlanguage{english}
\fi

\ifdefined\GERMAN
\wlog{main GERMAN defined OK}
\setmainlanguage{german}
\setotherlanguage{english}
\fi

\ifdefined\SPANISH
\setmainlanguage{spanish}
\setotherlanguage{english}
\fi

\ifdefined\ITALIAN
\setmainlanguage{italian}
\setotherlanguage{english}
\fi

\ifdefined\BRAZILIAN
\setmainlanguage{portuges}
\setotherlanguage{english}
\fi

\ifdefined\POLISH
\setmainlanguage{polish}
\setotherlanguage{english}
\fi

\ifdefined\FRENCH
\setmainlanguage{french}
\setotherlanguage{english}
\fi

\usepackage{microtype}
\usepackage{fancyhdr}
\usepackage{listings}
\usepackage{ulem}
\usepackage{url}
\usepackage{graphicx}
\usepackage{makeidx}
\usepackage[backend=biber,style=alphabetic]{biblatex}
%\usepackage{cite}
\usepackage[cm]{fullpage}
\usepackage{color}
\usepackage{fancyvrb}
\usepackage{xspace}
\usepackage{tabularx}
\usepackage{framed}
\usepackage{parskip}
\usepackage{epigraph}
\usepackage{ccicons}
\usepackage[nottoc]{tocbibind}
\usepackage{longtable}
\usepackage[footnote,printonlyused,withpage]{acronym}
\usepackage[table]{xcolor}% http://ctan.org/pkg/xcolor
\usepackage[]{bookmark,hyperref} % must be last
\usepackage[official]{eurosym}

% ************** myref
% http://tex.stackexchange.com/questions/228286/how-to-mix-ref-and-pageref#228292
\ifdefined\RUSSIAN
\newcommand{\myref}[1]{%
  \ref{#1} 
  (стр.~\pageref{#1})%
  }
% FIXME: I wasn't able to force varioref to output russian text...
\else
\usepackage{varioref}
\newcommand{\myref}[1]{\vref{#1}}
\fi
% ************** myref

\usepackage{glossaries}
\usepackage{tikz}
%\usepackage{fixltx2e}
\usepackage{bytefield}

\usepackage{amsmath}
\usepackage{MnSymbol}
\undef\mathdollar 

\usepackage{float}

\usepackage{shorttoc}
\usetikzlibrary{calc,positioning,chains,arrows}
\ifdefined\ebook
\usepackage[margin=0.5in,headheight=11pt]{geometry}
\else
\usepackage[margin=0.5in,headheight=12.5pt]{geometry}
\fi

\ifdefined\RUSSIAN
\renewcommand\lstlistingname{Листинг}
\renewcommand\lstlistlistingname{Листинг}
\fi

%\iffalse
% fancyhdr ********************************************************
\makeatletter
    \let\stdchapter\chapter
    \renewcommand*\chapter{%
    \@ifstar{\starchapter}{\@dblarg\nostarchapter}}
    \newcommand*\starchapter[1]{%
        \stdchapter*{#1}
        \thispagestyle{fancy}
        \markboth{\MakeUppercase{#1}}{}
    }
    \def\nostarchapter[#1]#2{%
        \stdchapter[{#1}]{#2}
        \thispagestyle{fancy}
    }
\makeatother

% taken from http://texblog.org/tag/fancyhdr-font-size/
\newcommand{\changefont}{%
\ifdefined\ebook
    \fontsize{6}{7}\selectfont
\else
    \fontsize{8}{9.5}\selectfont
\fi
}
\fancyhf{}
\fancyhead[L,RO]{\changefont \slshape \rightmark} %section
\fancyhead[R,LO]{\changefont \slshape \leftmark} %chapter
\fancyfoot[C]{\changefont \thepage} %footer
% *****************************************************************
%\fi

\newcommand{\footnoteref}[1]{\textsuperscript{\ref{#1}}}

\definecolor{lstbgcolor}{rgb}{0.94,0.94,0.94}
\makeindex

\newcommand*{\TT}[1]{\texttt{#1}}
\newcommand*{\IT}[1]{\textit{#1}}
\newcommand*{\EN}[1]{\iflanguage{english}{#1}{}}

\ifdefined\RUSSIAN{}
\newcommand*{\RU}[1]{\iflanguage{russian}{#1}{}}
\else
\newcommand*{\RU}[1]{}
\fi

\ifdefined\SPANISH{}
\newcommand*{\ES}[1]{\iflanguage{spanish}{#1}{}}
\else
\newcommand*{\ES}[1]{}
\fi

\ifdefined\ITALIAN{}
\newcommand*{\ITA}[1]{\iflanguage{italian}{#1}{}}
\else
\newcommand*{\ITA}[1]{}
\fi

\ifdefined\BRAZILIAN{}
\newcommand*{\PTBR}[1]{\iflanguage{portuges}{#1}{}}
\else
\newcommand*{\PTBR}[1]{}
\fi

\ifdefined\POLISH{}
\newcommand*{\PL}[1]{\iflanguage{polish}{#1}{}}
\else
\newcommand*{\PL}[1]{}
\fi

\ifdefined\GERMAN{}
%\newcommand*{\DE}[1]{\iflanguage{german}{#1}{}}
\newcommand*{\DE}[1]{#1}
\else
\newcommand*{\DE}[1]{}
\fi

\ifdefined\FRENCH{}
\newcommand*{\FR}[1]{\iflanguage{french}{#1}{}}
\else
\newcommand*{\FR}[1]{}
\fi

\newcommand{\LANG}{\RU{ru}\EN{en}\ES{es}\PTBR{ptbr}\PL{pl}\ITA{it}\DE{de}\FR{fr}}

\newcommand{\ESph}{\ES{Spanish text placeholder}}
\newcommand{\PTBRph}{\PTBR{Brazilian Portuguese text placeholder}}
\newcommand{\PLph}{\PL{Polish text placeholder}}
\newcommand{\ITAph}{\ITA{Italian text placeholder}}
\newcommand{\DEph}{\DE{German text placeholder}}
\newcommand{\FRph}{\FR{French text placeholder}}

\newcommand*{\dittoclosing}{---''---}
\newcommand*{\EMDASH}{\RU{~--- }\EN{---}\ESph{}\PTBRph{}\PLph{}\ITAph{}\DEph{}\FR{~--- }}
\newcommand*{\AsteriskOne}{${}^{*}$}
\newcommand*{\AsteriskTwo}{${}^{**}$}
\newcommand*{\AsteriskThree}{${}^{***}$}
\newcommand{\q}[1]{\enquote{#1}}
\newcommand{\var}[1]{\textit{#1}}

\newcommand{\ttf}{\TT{f()}\xspace}
\newcommand{\ttfone}{\TT{f1()}\xspace}

% without dot!
\newcommand{\etc}{%
	\RU{и~т.д}%
	\EN{etc}%
	\ES{etc}%
	\PTBRph{}%
	\PLph{}%
	\ITAph{}%
	\DEph{}%
	\FR{etc.}%
}

% http://tex.stackexchange.com/questions/32160/new-line-after-paragraph
\newcommand{\myparagraph}[1]{\paragraph{#1}\mbox{}\\} 

\newcommand{\figname}{%
	\RU{илл.}%
	\EN{fig.}%
	\ES{fig.}%
	\PTBRph{}%
	\PLph{}%
	\ITAph{}%
	\DEph{}%
	\FR{fig.}%
\xspace}
\newcommand{\figref}[1]{\figname{}\ref{#1}\xspace}
\newcommand{\listingname}{%
	\RU{листинг.}%
	\EN{listing.}%
	\ES{listado.}%
	\PTBRph{}%
	\PLph{}%
	\ITAph{}%
	\DEph{}%
	\FR{listing}%
\xspace}
\newcommand{\lstref}[1]{\listingname{}\ref{#1}\xspace}
\newcommand{\bitENRU}{%
	\RU{бит}%
	\EN{bit}%
	\ES{bit}%
	\PTBRph{}%
	\PLph{}%
	\ITAph{}%
	\DEph{}%
	\FR{bit}%
\xspace}
\newcommand{\bitsENRU}{%
	\RU{бита}%
	\EN{bits}%
	\ES{bits}%
	\PTBRph{}%
	\PLph{}%
	\ITAph{}%
	\DEph{}%
	\FR{bits}%
\xspace}
\newcommand{\Sourcecode}{%
	\RU{Исходный код}%
	\EN{Source code}%
	\ES{C\'odigo fuente}%
	\PTBRph{}%
	\PLph{}%
	\ITAph{}%
	\DEph{}%
	\FR{Code source}%
\xspace}
\newcommand{\Seealso}{%
	\RU{См. также}%
	\EN{See also}%
	\ES{V\'ease tambi\'en}%
	\PTBRph{}%
	\PLph{}%
	\ITAph{}%
	\DEph{}%
	\FR{Voir aussi}%
\xspace}
\newcommand{\MacOSX}{Mac OS X\xspace}

% FIXME TODO non-overlapping color!
% \newcommand{\headercolor}{\cellcolor{blue!25}}
\newcommand{\headercolor}{}

\newcommand{\tableheader}{\headercolor{}%
	\RU{смещение}%
	\EN{offset}%
	\ES{offset}%
	\PLph{}%
	\ITAph{}%
	\DEph{}%
	\FR{offset}%
& \headercolor{}%
	\RU{описание}%
	\EN{description}%
	\ES{descripci\'on}%
	\PTBRph{}%
	\PLph{}%
	\ITAph{}%
	\DEph{}%
	\FR{description}%
}

\newcommand{\IDA}{\ac{IDA}\xspace}

\newcommand{\tracer}{\protect\gls{tracer}\xspace}

\newcommand{\Tchar}{\IT{char}\xspace} 
\newcommand{\Tint}{\IT{int}\xspace}
\newcommand{\Tbool}{\IT{bool}\xspace}
\newcommand{\Tfloat}{\IT{float}\xspace}
\newcommand{\Tdouble}{\IT{double}\xspace}
\newcommand{\Tvoid}{\IT{void}\xspace}
\newcommand{\ITthis}{\IT{this}\xspace}

\newcommand{\Ox}{\TT{/Ox}\xspace}
\newcommand{\Obzero}{\TT{/Ob0}\xspace}
\newcommand{\Othree}{\TT{-O3}\xspace}

\newcommand{\oracle}{Oracle RDBMS\xspace}

\newcommand{\idevices}{iPod/iPhone/iPad\xspace}
\newcommand{\olly}{OllyDbg\xspace}

% common C functions
\newcommand{\printf}{\TT{printf()}\xspace} 
\newcommand{\puts}{\TT{puts()}\xspace} 
\newcommand{\main}{\TT{main()}\xspace} 
\newcommand{\qsort}{\TT{qsort()}\xspace} 
\newcommand{\strlen}{\TT{strlen()}\xspace} 
\newcommand{\scanf}{\TT{scanf()}\xspace} 
\newcommand{\rand}{\TT{rand()}\xspace} 


% for easier fiddling with formatting of all instructions together
\newcommand{\INS}[1]{\TT{#1}\xspace}

% x86 instructions
\newcommand{\ADD}{\INS{ADD}}
\newcommand{\ADRP}{\INS{ADRP}}
\newcommand{\AND}{\INS{AND}}
\newcommand{\CALL}{\INS{CALL}}
\newcommand{\CPUID}{\INS{CPUID}}
\newcommand{\CMP}{\INS{CMP}}
\newcommand{\DEC}{\INS{DEC}}
\newcommand{\FADDP}{\INS{FADDP}}
\newcommand{\FCOM}{\INS{FCOM}}
\newcommand{\FCOMP}{\INS{FCOMP}}
\newcommand{\FCOMI}{\INS{FCOMI}}
\newcommand{\FCOMIP}{\INS{FCOMIP}}
\newcommand{\FUCOM}{\INS{FUCOM}}
\newcommand{\FUCOMI}{\INS{FUCOMI}}
\newcommand{\FUCOMIP}{\INS{FUCOMIP}}
\newcommand{\FUCOMPP}{\INS{FUCOMPP}}
\newcommand{\FDIVR}{\INS{FDIVR}}
\newcommand{\FDIV}{\INS{FDIV}}
\newcommand{\FLD}{\INS{FLD}}
\newcommand{\FMUL}{\INS{FMUL}}
\newcommand{\MUL}{\INS{MUL}}
\newcommand{\FSTP}{\INS{FSTP}}
\newcommand{\FDIVP}{\INS{FDIVP}}
\newcommand{\IDIV}{\INS{IDIV}}
\newcommand{\IMUL}{\INS{IMUL}}
\newcommand{\INC}{\INS{INC}}
\newcommand{\JAE}{\INS{JAE}}
\newcommand{\JA}{\INS{JA}}
\newcommand{\JBE}{\INS{JBE}}
\newcommand{\JB}{\INS{JB}}
\newcommand{\JE}{\INS{JE}}
\newcommand{\JGE}{\INS{JGE}}
\newcommand{\JG}{\INS{JG}}
\newcommand{\JLE}{\INS{JLE}}
\newcommand{\JL}{\INS{JL}}
\newcommand{\JMP}{\INS{JMP}}
\newcommand{\JNE}{\INS{JNE}}
\newcommand{\JNZ}{\INS{JNZ}}
\newcommand{\JNA}{\INS{JNA}}
\newcommand{\JNAE}{\INS{JNAE}}
\newcommand{\JNB}{\INS{JNB}}
\newcommand{\JNBE}{\INS{JNBE}}
\newcommand{\JZ}{\INS{JZ}}
\newcommand{\JP}{\INS{JP}}
\newcommand{\Jcc}{\INS{Jcc}}
\newcommand{\SETcc}{\INS{SETcc}}
\newcommand{\LEA}{\INS{LEA}}
\newcommand{\LOOP}{\INS{LOOP}}
\newcommand{\MOVSX}{\INS{MOVSX}}
\newcommand{\MOVZX}{\INS{MOVZX}}
\newcommand{\MOV}{\INS{MOV}}
\newcommand{\NOP}{\INS{NOP}}
\newcommand{\POP}{\INS{POP}}
\newcommand{\PUSH}{\INS{PUSH}}
\newcommand{\NOT}{\INS{NOT}}
\newcommand{\NOR}{\INS{NOR}}
\newcommand{\RET}{\INS{RET}}
\newcommand{\RETN}{\INS{RETN}}
\newcommand{\SETNZ}{\INS{SETNZ}}
\newcommand{\SETBE}{\INS{SETBE}}
\newcommand{\SETNBE}{\INS{SETNBE}}
\newcommand{\SUB}{\INS{SUB}}
\newcommand{\TEST}{\INS{TEST}}
\newcommand{\TST}{\INS{TST}}
\newcommand{\FNSTSW}{\INS{FNSTSW}}
\newcommand{\SAHF}{\INS{SAHF}}
\newcommand{\XOR}{\INS{XOR}}
\newcommand{\OR}{\INS{OR}}
\newcommand{\SHL}{\INS{SHL}}
\newcommand{\SHR}{\INS{SHR}}
\newcommand{\SAR}{\INS{SAR}}
\newcommand{\LEAVE}{\INS{LEAVE}}
\newcommand{\MOVDQA}{\INS{MOVDQA}}
\newcommand{\MOVDQU}{\INS{MOVDQU}}
\newcommand{\PADDD}{\INS{PADDD}}
\newcommand{\PCMPEQB}{\INS{PCMPEQB}}
\newcommand{\LDR}{\INS{LDR}}
\newcommand{\LSL}{\INS{LSL}}
\newcommand{\LSR}{\INS{LSR}}
\newcommand{\ASR}{\INS{ASR}}
\newcommand{\RSB}{\INS{RSB}}
\newcommand{\BTR}{\INS{BTR}}
\newcommand{\BTS}{\INS{BTS}}
\newcommand{\BTC}{\INS{BTC}}
\newcommand{\LUI}{\INS{LUI}}
\newcommand{\ORI}{\INS{ORI}}
\newcommand{\BIC}{\INS{BIC}}
\newcommand{\EOR}{\INS{EOR}}
\newcommand{\MOVS}{\INS{MOVS}}
\newcommand{\LSLS}{\INS{LSLS}}
\newcommand{\LSRS}{\INS{LSRS}}
\newcommand{\FMRS}{\INS{FMRS}}
\newcommand{\CMOVNE}{\INS{CMOVNE}}
\newcommand{\CMOVNZ}{\INS{CMOVNZ}}
\newcommand{\ROL}{\INS{ROL}}
\newcommand{\CSEL}{\INS{CSEL}}
\newcommand{\SLL}{\INS{SLL}}
\newcommand{\SLLV}{\INS{SLLV}}
\newcommand{\SW}{\INS{SW}}
\newcommand{\LW}{\INS{LW}}

% x86 flags

\newcommand{\ZF}{\TT{ZF}\xspace} 
\newcommand{\CF}{\TT{CF}\xspace} 
\newcommand{\PF}{\TT{PF}\xspace} 

% x86 registers

\newcommand{\AL}{\TT{AL}\xspace} 
\newcommand{\AH}{\TT{AH}\xspace} 
\newcommand{\AX}{\TT{AX}\xspace} 
\newcommand{\EAX}{\TT{EAX}\xspace} 
\newcommand{\EBX}{\TT{EBX}\xspace} 
\newcommand{\ECX}{\TT{ECX}\xspace} 
\newcommand{\EDX}{\TT{EDX}\xspace} 
\newcommand{\DL}{\TT{DL}\xspace} 
\newcommand{\ESI}{\TT{ESI}\xspace} 
\newcommand{\EDI}{\TT{EDI}\xspace} 
\newcommand{\EBP}{\TT{EBP}\xspace} 
\newcommand{\ESP}{\TT{ESP}\xspace} 
\newcommand{\RSP}{\TT{RSP}\xspace} 
\newcommand{\EIP}{\TT{EIP}\xspace} 
\newcommand{\RIP}{\TT{RIP}\xspace} 
\newcommand{\RAX}{\TT{RAX}\xspace} 
\newcommand{\RBX}{\TT{RBX}\xspace} 
\newcommand{\RCX}{\TT{RCX}\xspace} 
\newcommand{\RDX}{\TT{RDX}\xspace} 
\newcommand{\RBP}{\TT{RBP}\xspace} 
\newcommand{\RSI}{\TT{RSI}\xspace} 
\newcommand{\RDI}{\TT{RDI}\xspace} 
\newcommand*{\ST}[1]{\TT{ST(#1)}\xspace}
\newcommand*{\XMM}[1]{\TT{XMM#1}\xspace}

% ARM
\newcommand*{\Reg}[1]{\TT{R#1}\xspace}
\newcommand*{\RegX}[1]{\TT{X#1}\xspace}
\newcommand*{\RegW}[1]{\TT{W#1}\xspace}
\newcommand*{\RegD}[1]{\TT{D#1}\xspace}
\newcommand{\ADREQ}{\TT{ADREQ}\xspace}
\newcommand{\ADRNE}{\TT{ADRNE}\xspace}
\newcommand{\BEQ}{\TT{BEQ}\xspace}

% instructions descriptions
\newcommand{\ASRdesc}{%
	\RU{арифметический сдвиг вправо}%
	\EN{arithmetic shift right}%
	\ES{desplazamiento aritm\'etico a la derecha}%
	\PTBRph{}%
	\PLph{}%
	\ITAph{}%
	\DEph{}%
	\FR{décalage arithmétique}%
}

% x86 registers tables
% TODO: non-overlapping color!
\newcommand{\RegHeader}{
\RU{
 7 \textsuperscript{(номер байта)} &
 6 &
 5 &
 4 &
 3 &
 2 &
 1 &
 0 }%
\EN{
 7th \textsuperscript{(byte number)} &
 6th &
 5th &
 4th &
 3rd &
 2nd &
 1st &
 0th}%
\ES{
 7mo \textsuperscript{(n\'umero de byte)} &
 6to &
 5to &
 4to &
 3ro &
 2do &
 1ro &
 0}%
\PTBRph{}%
\PLph{}%
\ITAph{}%
\DEph{}%
\FR{
 7\textsuperscript{e} \textsuperscript{numéro d'octet} &
 6\textsuperscript{e} &
 5\textsuperscript{e} &
 4\textsuperscript{e} &
 3\textsuperscript{e} &
 2\textsuperscript{e} &
 1\textsuperscript{er} &
 0\textsuperscript{e} &
 }%
}

% FIXME навести порядок тут...
\newcommand{\RegTableThree}[5]{
\begin{center}
\begin{tabular}{ | l | l | l | l | l | l | l | l | l |}
\hline
\RegHeader \\
\hline
\multicolumn{8}{ | c | }{#1} \\
\hline
\multicolumn{4}{ | c | }{} & \multicolumn{4}{ c | }{#2} \\
\hline
\multicolumn{6}{ | c | }{} & \multicolumn{2}{ c | }{#3} \\
\hline
\multicolumn{6}{ | c | }{} & #4 & #5 \\
\hline
\end{tabular}
\end{center}
}

\newcommand{\RegTableOne}[5]{\RegTableThree{#1\textsuperscript{x64}}{#2}{#3}{#4}{#5}}

\newcommand{\RegTableTwo}[4]{
\begin{center}
\begin{tabular}{ | l | l | l | l | l | l | l | l | l |}
\hline
\RegHeader \\
\hline
\multicolumn{8}{ | c | }{#1\textsuperscript{x64}} \\
\hline
\multicolumn{4}{ | c | }{} & \multicolumn{4}{ c | }{#2} \\
\hline
\multicolumn{6}{ | c | }{} & \multicolumn{2}{ c | }{#3} \\
\hline
\multicolumn{7}{ | c | }{} & #4\textsuperscript{x64} \\
\hline
\end{tabular}
\end{center}
}

\newcommand{\RegTableFour}[4]{
\begin{center}
\begin{tabular}{ | l | l | l | l | l | l | l | l | l |}
\hline
\RegHeader \\
\hline
\multicolumn{8}{ | c | }{#1} \\
\hline
\multicolumn{4}{ | c | }{} & \multicolumn{4}{ c | }{#2} \\
\hline
\multicolumn{6}{ | c | }{} & \multicolumn{2}{ c | }{#3} \\
\hline
\multicolumn{7}{ | c | }{} & #4 \\
\hline
\end{tabular}
\end{center}
}


\include{glossary}
\include{common_URLS}
\include{common_phrases}

\makeglossaries

\newcommand{\TITLE}{%
\RU{Reverse Engineering для начинающих}%
\EN{Reverse Engineering for Beginners}%
\ES{Ingenier\'ia Inversa para Principiantes}%
\PTBRph{}%
\DEph{}\PLph{}%
\ITAph{}%
\FR{La Rétro-Ingénierie pour les Débutants}%
}
\newcommand{\AUTHOR}{%
\RU{Денис Юричев}%
\EN{Dennis Yurichev}%
\ES{Dennis Yurichev}%
\PTBRph{}%
\DEph{}\PLph{}%
\ITAph{}%
\FR{Dennis Yurichev}%
}
\newcommand{\EMAIL}{dennis(a)yurichev.com}

\hypersetup{
    colorlinks=true,
    allcolors=blue,
    pdfauthor={\AUTHOR},
    pdftitle={\TITLE}
    }

%\ifdefined\RUSSIAN
\newcommand{\LstStyle}{\ttfamily\small}
%\else
%\newcommand{\LstStyle}{\ttfamily}
%\fi 

\lstset{
    backgroundcolor=\color{lstbgcolor},
    basicstyle=\LstStyle,
    breaklines=true,
    %prebreak=\raisebox{0ex}[0ex][0ex]{->},
    %postbreak=\raisebox{0ex}[0ex][0ex]{->},
    prebreak=\raisebox{0ex}[0ex][0ex]{\ensuremath{\rhookswarrow}},
    postbreak=\raisebox{0ex}[0ex][0ex]{\ensuremath{\rcurvearrowse\space}},
    frame=single,
    columns=fullflexible,keepspaces,
    escapeinside=§§,
    inputencoding=utf8
}

\DeclareMathSizes{12}{30}{16}{12}%
\RU{\bibliography{C_book_ru,books,articles,usenet,misc}}%
\EN{\bibliography{C_book_en,books,articles,usenet,misc}}%
\ES{\bibliography{C_book_en,books,articles,usenet,misc}}%
\PTBR{\bibliography{C_book_en,books,articles,usenet,misc}}%
\DE{\bibliography{C_book_en,books,articles,usenet,misc}}%
\PL{\bibliography{C_book_en,books,articles,usenet,misc}}%
\ITA{\bibliography{C_book_en,books,articles,usenet,misc}}%
\FR{\bibliography{C_book_fr,books,articles,usenet,misc}}%

\begin{document}

\pagestyle{fancy}

\VerbatimFootnotes

\frontmatter

\begin{titlepage}

\input{cover}
\end{titlepage}

\newpage

\begin{center}
\vspace*{\fill}
{\LARGE \TITLE}

\vspace*{\fill}

{\large \AUTHOR}

{\large \TT{<\EMAIL>}}
\vspace*{\fill}
\vfill

\ccbyncnd

\textcopyright 2013-2015, \AUTHOR. 

\RU{Это произведение доступно по лицензии Creative Commons «Attribution-NonCommercial-NoDerivs» 
(«Атрибуция~—~Некоммерческое использование~—~Без производных произведений») 3.0 Непортированная. 
Чтобы увидеть копию этой лицензии, посетите}%
\EN{This work is licensed under the Creative Commons Attribution-NonCommercial-NoDerivs 3.0 Unported License. 
To view a copy of this license, visit}%
\ES{Esta obra est\'a bajo una Licencia Creative Commons Atribuci\'on-NoComercial-SinDerivadas 3.0 Unported
Para ver una copia de esta licencia, visita}%
\PTBRph{}%
\DEph{}\PLph{}%
\ITAph{}%
    \FR{Ce travail est sous licence Creative Commons «~Attribution-NonCommercial-NoDerivs~» («~Attribution~—~Pas d'utilisation commerciale~—~Pas de modification~»).
Pour voir une copie de cette licence, visiter}
\url{http://creativecommons.org/licenses/by-nc-nd/3.0/}.

\RU{Версия этого текста}%
\EN{Text version}%
\ES{Versi\'on del texto}%
\PTBRph{}%
\DEph{}\PLph{}%
\ITAph{}
({\large \today}).

\RU{Самая новая версия текста (а также англоязычная версия) доступна на сайте}%
\EN{The latest version (and Russian edition) of this text is accessible at}%
\ES{La \'ultima versi\'on (as\'i como las versiones en ingl\'es y ruso) de este texto est\'a disponible en}%
\PTBRph{}%
\DEph{}\PLph{}%
\ITAph{}%
\FR{La dernière version (et les traductions) de ce texte sont disponibles sur}
\href{http://go.yurichev.com/17009}{beginners.re}.
\ifdefined\ebook
\RU{Версия формата A4 доступна там же.}
\EN{An A4-format version is also available.}
\ES{Una versi\'on en formato A4 tambi\'en est\'a disponible.}%
\PTBRph{}%
\DEph{}\PLph{}%
\ITAph{}%
\FR{Une version au format A4 y est aussi disponible.}
\else
\RU{Версия для электронных читалок так же доступна на сайте.}%
\EN{An e-book reader version is also available.}%
\ES{Una versi\'on para lector de libros electr\'onicos tambi\'en est\'a disponible.}%
\PTBRph{}%
\DEph{}\PLph{}%
\ITAph{}%
\FR{Une version pour les lecteurs de livres électroniques y est aussi disponible.}
\fi

\ifx\LITE\undefined
\RU{Существует также LITE-версия (сокращенная вводная версия), предназначенная для быстрого ознакомления
с основами reverse engineering:}%
\EN{There is also a LITE-version (introductory short version), intended for those who want a 
very quick introduction to the basics of reverse engineering:}%
\ES{Tambi\'en existe una versi\'on LITE (versi\'on corta e introductoria), dirigida a aquellos
que busquen una introducci\'on r\'apida a los fundamentos de la ingenier\'ia inversa:}%
\PTBRph{}%
\DEph{}\PLph{}%
\ITAph{}%
\FR{Il existe aussi une version LITE (version courte d'introduction), destinée à ceux qui veulent
une très brève introduction sur les bases de la rétro-ingénierie:}
\href{http://go.yurichev.com/17358}{beginners.re}
\fi

\RU{Вы также можете подписаться на мой twitter для получения информации о новых версиях этого текста:}%
\EN{You can also follow me on twitter to get information about updates of this text:}%
\ES{Adem\'as puedes seguirme en twitter para obtener informaci\'on sobre actualizaciones de este texto:}%
\PTBRph{}%
\DEph{}\PLph{}%
\ITAph{}%
\FR{Vous pouvez aussi me suivre sur twitter pour obtenir des informations à propos des mises à jour de ce texte:}
\TT{@yurichev}\footnote{\href{http://go.yurichev.com/17021}{twitter.com/yurichev}},
\RU{либо подписаться на список рассылки}%
\EN{or subscribe to the mailing list}%
\ES{o subscribirte a la lista de correo}%
\PTBRph{}%
\DEph{}\PLph{}%
\ITAph{}%
\FR{ou souscrire à la liste de distribution}%
\footnote{\href{http://go.yurichev.com/17020}{yurichev.com}}.

\RU{Обложка нарисована Андреем Нечаевским:}%
\EN{The cover was made by Andy Nechaevsky:}%
\ES{La portada fue hecha por Andy Nechaevsky:}%
\PTBRph{}%
\DEph{}\PLph{}%
\ITAph{}%
\FR{La couverture a été réalisée par Andy Nechaevsky:}
\href{http://go.yurichev.com/17023}{facebook}.

\end{center}

\vspace*{\fill}

\Huge%
	\RU{Нужны переводчики!}%
	\EN{Call for translators!}%
	\ESph{}%
	\PTBRph{}%
	\DEph{}%
	\PLph{}%
	\ITAph{}%
	\FR{Appel aux traducteurs!}
\normalsize

\bigskip
\bigskip
\bigskip

\EN{You may want to help me with translation this work into languages other than English and Russian.}%
\RU{Возможно, вы захотите мне помочь с переводом этой работы на другие языки, кроме английского и русского.}%
\FR{Vous pourriez vouloir m'aider pour la traduction de ce travail dans d'autres langues que l'anglais et le russe.}

\EN{For those who are not afraid of TeX: \href{https://github.com/dennis714/RE-for-beginners/blob/master/Translation.md}{read here}.}%
\RU{Для тех, кто не боится TeX: \href{https://github.com/dennis714/RE-for-beginners/blob/master/Translation.md}{читайте здесь}.}%
\FR{Pour ceux qui n'ont pas peur de \TeX~: \href{https://github.com/dennis714/RE-for-beginners/blob/master/Translation.md}{lisez ici}.}
\EN{For those who afraid, you may just open PDF file in OpenOffice and gradually rewrite each sentence.}%
\RU{Для тех, кто боится, вы можете просто открыть PDF-файл в OpenOffice и постепенно переписывать каждое предложение.}%
\FR{Pour ceux qui en ont peur, vous pouvez simplement ouvrir le fichier PDF dans OpenOffice et réécrire}
\EN{I'll copypaste your work back to my LaTeX source code.}%
\RU{Я затем скопирую вашу работу назад в исходный код на LaTeX.}%
\FR{Je copierai votre travail dans mon code source \LaTeX.}

\EN{It's a tedious and boring work, so you probably may want to start with shortened \href{http://beginners.re/\#lite}{LITE version}.}%
\RU{Это рутинная и скучная работа, так что вы можете начать с сокращенной \href{http://beginners.re/\#lite}{LITE-версии}.}%
\FR{C'est un travail fastidieux et ennuyeux, donc vous devriez probablement commencer avec la \href{http://beginners.re/\#lite}{version LITE} résumée.}
\EN{There is even a better way: to my own experience, you can gain your motivation by translating short pieces of my book and posting them to your blog(s).}%
\RU{Есть даже еще лучше способ: по моему опыту, мотивировать себя можно переводя короткие части текста из моей книги и выкладывая их в своем блоге.}%
\FR{Il y a une encore meilleure façon~: d'après mon expérience, vous pouvez augmenter votre motivation en traduisant de courts passages de mon livre et en les postant sur votre/vos blog(s).}
\EN{I can publish URLs to these your posts here and also in my twitter (\href{http://twitter.com/yurichev}{@yurichev}).}%
\RU{Я могу публиковать URL-ы на ваши посты здесь, а также в моем twitter (\href{http://twitter.com/yurichev}{@yurichev}).}%
\FR{Je peux publier les URLs vers vos articles et aussi sur mon twitter (\href{http://twitter.com/yurichev}{@yurichev}).}

\EN{Speed isn't important, because this is open-source project, after all.}%
\RU{Скорость не важна, потому что это опен-сорсный проект все-таки.}%
\FR{La vitesse n'a pas d'importance, parce que c'est un projet open-source, après tout.}
\EN{Your name will be mentioned as project contributor.}%
\RU{Ваше имя будет указано в числе участников проекта.}%
\FR{Votre nom sera mentionné comme contributeur au projet.}

\EN{Korean, Chinese and Persian languages are reserved by publishers.}%
\RU{Корейский, китайский и персидский языки зарезервированы издателями.}%
\FR{Le coréen, le chinois et le persan sont réservés par des éditeurs.}
\EN{As of March 2016, there are Brazilian Portuguese and German language teams working, drop me email, so I will connect you to them.}%
\RU{По состоянию на март 2016, есть две команды (бразильский португальский и немецкий), напишите мне, и я соеденю вас с ними.}%
\FR{À compter de mars 2016, il y a des équipes travaillant sur le portugais brésilien et l'allemand, envoyez-moi un e-mail, que je vous mette en contact avec eux.}
\EN{All other attempts to translate pieces of these texts to other languages are highly welcomed!}%
\RU{Все остальные попытки перевести части этих текстов на другие языки очень приветствуются!}%
\FR{Toutes les autres tentatives de traduire des morceaux de ces textes dans d'autres langues sont plus que bienvenues~!}

\EN{English and Russian versions I do by myself, but my English is still that horrible, so I'm very grateful for any notes about grammar, etc.}%
\RU{Английскую и русскую версии я делаю сам, но английский у меня все еще ужасный, так что я буду очень признателен за коррективы, итд.}%
\FR{Je m'occupe des versions anglaise et russe moi-même, mais mon anglais est toujours tellement horrible que je vous remercie pour n'importe quel commentaire à propos de la grammaire, etc.}
\EN{Even my Russian is also flawed, so I'm grateful for notes about Russian text as well!}%
\RU{Даже мой русский несовершенный, так что я благодарен за коррективы и русского текста!}%
\FR{Même mon russe est imparfait, donc je suis reconnaissant pour des commentaires à propos du texte russe aussi~!}

\EN{So do not hesitate to contact me: \TT{\EMAIL}.}%
\RU{Не стесняйтесь писать мне: \TT{\EMAIL}.}%
\FR{Donc n'hésitez pas à me contacter~: \TT{\EMAIL}.}

\vspace*{\fill}
\vfill

\ifdefined\LITE
\include{LITE_warning}
\fi
%\include{survey}

\ifx\LITE\undefined
\shorttoc{%
	\RU{Краткое оглавление}%
	\EN{Abridged contents}%
	\ES{Contenidos abreviados}%
	\PTBRph{}%
	\DEph{}\PLph{}%
	\ITAph{}%
    \FR{Table des matières}%
}{-1} % Only sections
\fi
\tableofcontents
\cleardoublepage

\cleardoublepage
\include{preface}

\mainmatter

\ifx\LITE\undefined
\include{contents}
\fi

\include{parts}

\ifdefined\LITE
\include{LITE_warning}
\fi

\ifx\LITE\undefined
\include{appendix/appendix}
\fi
\include{acronyms}

\bookmarksetup{startatroot}

\clearpage
\phantomsection
\addcontentsline{toc}{chapter}{%
	\RU{Глоссарий}%
	\EN{Glossary}%
	\ES{Glosario}%
	\PTBRph{}%
	\DEph{}\PLph{}%
	\ITAph{}%
	\FR{Glossaire}%
}
\printglossaries

\clearpage
\phantomsection
\printindex

\clearpage
\phantomsection
\addcontentsline{toc}{chapter}{%
	\RU{Библиография}%
	\EN{Bibliography}%
	\ES{Bibliograf\'ia}%
	\PTBRph{}%
	\DEph{}\PLph{}%
	\ITAph{}%
    \FR{Bibliographie}%
}
\printbibliography

\end{document}
